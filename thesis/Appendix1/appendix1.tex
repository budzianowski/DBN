% ******************************* Thesis Appendix A ****************************
\chapter{EMF derivation} 

Following \cite{georges1991expand} lets define
energy as:
\begin{align}
\begin{split}
E = -\sum_{ij} w_{ij}s_i s_j - \sum_i \theta_i s_i
\end{split}
\end{align}
and introduce the following operator:
\begin{align}
\begin{split}
U \equiv E - \mathbb{E}(E) - \sum_i \frac{\partial \lambda_i (\beta)}{\partial \beta} (s_i - m_i)
\label{eq:Uoperator}
\end{split}
\end{align}
which poses useful property -- $\mathbb{E}(U) = 0$. For any other operator $O$ we then have:
\begin{align}
\begin{split}
 \frac{\partial \mathbb{E}(O)}{\partial \beta}  =   \mathbb{E} \left(\frac{\partial O}{\partial \beta} \right) - \mathbb{E}(OU).
 \label{eq:obeta}
\end{split}
\end{align}
Now, the first derivative from the Taylor expansion is:
\begin{align*}
\begin{split}
\frac{\partial (\beta F)}{\partial \beta} = &
\dfrac{\sum_{\mathbf{s}} \exp \left( \beta \sum_{(ij)} w_{ij} s_i s_j +  \beta \sum_i \theta_i s_i+ \sum_i \lambda_i (\beta) (s_i - m_i) \right)  }
{\sum_{\mathbf{s}} \exp \left( \beta \sum_{(ij)} w_{ij} s_i s_j + \sum_i \theta_i s_i + \sum_i \lambda_i (\beta) (s_i - m_i) \right)} \\
& \cdot \left(\sum_{(ij)} w_{ij} s_i s_j +  \sum_i \theta_i s_i + \sum_i \frac{\partial \lambda_i(\beta)}{\partial \beta} (s_i - m_i) \right) \\
 = & \sum_{(ij)} w_{ij} \mathbb{E}(  s_i s_j) +  \sum_i a_i  \mathbb{E}( s_i) + \frac{\partial \lambda_i(\beta)}{\partial \beta} \sum_i \mathbb{E}(
 s_i - m_i ).
\end{split}
\end{align*}
In the case of $\beta = 0$ we have:
$$ \left. \frac{\partial (\beta F)}{\partial \beta}\right|_{\beta = 0} = \sum_{(ij)} w_{ij} m_i m_j +  \sum_i \theta_i m_i .$$
Using \ref{eq:obeta} we obtain:
\begin{align}
\begin{split}
\frac{\partial m_i}{\partial \beta} = 0 = \frac{\partial \mathbb{E}(s_i)}{\partial \beta}  =   \mathbb{E} \left(\frac{\partial s_i}{\partial \beta} \right) - \mathbb{E}(s_iU) =- \mathbb{E}(s_iU)  =- \mathbb{E}(U(s_i - m_i)).
\end{split}
\end{align}
The first derivative of the operator $U$ has the form:
\begin{align}
\begin{split}
\frac{\partial U}{\partial \beta}  =  ~&\frac{\partial E}{\partial \beta} - \frac{\partial \mathbb{E}(E)}{\partial \beta} - \sum_i \frac{\partial^2 \lambda_i (\beta)}{\partial \beta^2} (s_i - m_i)\\
= ~& \mathbb{E}(U^2) - \sum_i \frac{\partial^2 \lambda_i (\beta)}{\partial \beta^2} (s_i - m_i)
\end{split}
\end{align}
and the second derivative is:
\begin{align}
\begin{split}
\frac{\partial^2 U}{\partial \beta^2}  =  ~& 2 \mathbb{E}\left(\frac{\partial U}{\partial \beta}U  \right) - \mathbb{E}(U^3) - \sum_i \frac{\partial^3 \lambda_i (\beta)}{\partial \beta^3} (s_i - m_i) \\
= ~& - \mathbb{E}(U^3) - \sum_i \frac{\partial^3 \lambda_i (\beta)}{\partial \beta^3} (s_i - m_i)
\end{split}
\end{align}
The expansion of free energy using formulas derived above can now be reformulated in terms of the operator $U$:
\begin{align}
\begin{split}
 \frac{\partial (\beta F)}{\partial \beta} = \mathbb{E}(E) - \sum_i \frac{\partial\lambda_i (\beta)}{\partial \beta}\mathbb{E}(s_i - m_i) = \mathbb{E}(E) 
\end{split}
\label{eq:naive}
\end{align}
and the higher orders:
\begin{align}
\begin{split}
 \frac{\partial^2 (\beta F)}{\partial \beta^2} = &~ \mathbb{E}\left( \frac{\partial E}{\partial \beta} \right) - \mathbb{E}(EU) = - \mathbb{E}(U^2),\\
 \frac{\partial^3 (\beta F)}{\partial \beta^3}= &~ - 2\mathbb{E}\left( U \frac{\partial U}{\partial \beta} \right) + \mathbb{E}(U^3)  = \mathbb{E}(U^3).
  \label{eq:higherOrders}
\end{split}
\end{align}
Taylor expansion was considered around point $\beta =0$. Using derivations from \ref{eq:naive} we obtain again a 'naive' term:
\begin{align}
\begin{split}
\left. \frac{\partial (\beta F)}{\partial \beta} \right|_{\beta =0} = \mathbb{E} _{\beta = 0}(E) = - \sum_{(ij)} w_{ij} m_i m_j - \sum_i \theta_i m_i = - \frac{1}{2} \sum_i \sum_j w_{ij}m_i m_j - \sum_i \theta_i m_i.
\end{split}
\end{align}
Consider now:
\begin{align}
\begin{split}
\left.\frac{\partial (\beta F)}{\partial m_i \partial \beta}\right|_{\beta = 0} = 
- \sum_{j \neq i} w_{ij} m_j - \theta_i.
\end{split}
\end{align}
On the other hand:
\begin{align}
\begin{split}
\left. \frac{\partial (\beta F)}{\partial m_i \partial \beta}\right|_{\beta = 0} = \left. \frac{\partial (\beta F)}{\partial \beta \partial m_i }\right|_{\beta = 0} = \frac{\partial}{\partial \beta} \mathbb{E} (\lambda_i(\beta)) =  \left. \frac{\partial \lambda_i (\beta)}{\partial \beta}\right|_{\beta =0}
 \label{eq:maxwell}
 \end{split}
\end{align}
Substituting  \ref{eq:maxwell} into \ref{eq:Uoperator} gives us:
\begin{align}
\begin{split}
U_{\beta = 0} = & -\sum_{(ij)} w_{ij}s_is_j -\sum_i \theta_i s_i + \frac{1}{2} \sum_i \sum_j w_{ij} m_i m_j + \sum_i \theta_i m_i + \sum_i \left(\sum_{j \neq i} w_{ij} m_j  + \theta_i \right)(s_i - m_i ) \\
= & -\sum_{(ij)} w_{ij}s_i s_j - \frac{1}{2}\sum_{(ij)} w_{ij} m_i m_j + \sum_i \sum_{j \neq i} w_{ij} s_i m_j \\
= & - \sum_{(ij)} w_{ij} (s_i - m_i)(s_j- m_j) = - \sum_l w_l y_l
 \end{split}
\end{align}
where $w_l = w_{ij}$ and $y_l = (s_i -m_i)(s_j-m_j)$ stands for the 'link' operator which poses useful properties:
\begin{align}
\begin{split}
\mathbb{E}(y_l)_{\beta =0} = & ~\mathbb{E}(s_i s_j) -m_j\mathbb{E}(s_i) - m_i\mathbb{E}(s_j) + m_i m_j = 0 \\
\mathbb{E}(y_l(s_i-m_i))_{\beta =0} = & ~ m_j - m_j -m_i^2m_j + m_i^2m_j \\
 & ~ - m_i^2m_j + m_i^2m_j + m_i^2m_j - m_i^2m_j \\
 = &~ 0.
 \end{split}
\end{align}
Finally, if $k \neq l$ then:
$$\mathbb{E}(y_k y_l)= \mathbb{E}(y_k)\mathbb{E}(y_l)=0$$
while for $k = l$ we have:
\begin{align}
\begin{split}
\mathbb{E}((s_i-m_1)^2(s_j-m_j)^2)= & ~m_im_j - 2m_im_j^2 +m_im_j^2 - 2m_i^2m_j + 4m_1^2m_j^2\\
& - 2m_i^2m_j^2 + m_i^2m_j -2m_i^2m_j^2 + m_i^2m_j^2 \\
= & ~ (m_i -m_i^2)(m_j-m_j^2).
\label{eq:Yoperator}
 \end{split}
\end{align}
Using properties from $y_l$ in equations \ref{eq:higherOrders} we can derive:
\begin{align*}
\begin{split}
\left. \frac{\partial^2 (\beta F)}{\partial \beta^2}\right|_{\beta = 0} = & -\mathbb{E}(U^2)_{\beta =0}\\
= & - \sum_{l_i l_2} w_{l_i}w_{l_2} \mathbb{E}_{\beta = 0} (y_{l_1}y_{l_2} ) \\
= & - \sum_{(i,j)} w_{ij}^2 (m_i-m_i^2)(m_j-m_j^2)
\end{split}
\end{align*}
which yields the TAP-Onsager term. 
%ing approach from \ref{eq:maxwell} we have:
%\begin{align*}
%\begin{split}
%\left.\frac{\partial^2 \lambda_i (\beta)}{\partial \beta^2}\right|_{\beta =0} = \left.\frac{\partial^3 (\beta G)}{\partial m_i \partial \beta^2}\right|_{\beta = 0} = 2m_i \sum_{j \neq i} w_{ij}^2(1-m_j^2)
%\end{split}
%\end{align*}
To obtain the next term in the Taylor expansion we need to compute $\mathbb{E}(y_{l_1} y_{l_2} y_{l_3})$ term and by definition the structure of the RBM model doesn't admit triangles in its corresponding factor graphs. Thus, we need to consider only the case when $l_1 = l_2 = l_3$:
\begin{align}
\begin{split}
\mathbb{E}((s_i-m_1)^3(s_j-m_j)^3)= & ~m_i m_j -3m_i m_j^2 +2 m_i m_j^2 + 2m_im_j^3 -3 m_i^2 m_j 
 + 2 m_i^3 m_j \\
 & + 9 m_i^2m_j^2 - 6m_i^3 m_j^2 - 6 m_i^2 m_j^3 + 4m_i^3m_j^3 
 \\
  = &~ 4(m_i - m_i^2)(\frac{1}{2} - m_i)(m_j - m_j^2)(\frac{1}{2} - m_j)
 \end{split}
\end{align}
and the third-order term in the case of the RBM structure:
\begin{align*}
\begin{split}
\left. \frac{\partial^3 (\beta F)}{\partial \beta^3}\right|_{\beta = 0} = \frac{2\beta^3}{3} \sum_{(ij)} w_{ij}^3 (m_i - m_i^2)(\frac{1}{2} - m_i)(m_j - m_j^2)(\frac{1}{2} - m_j).
\end{split}
\end{align*}


